\documentclass[signature=preparation]{physicsreport}

%%
%% User settings
%%

\classno{}
\stuno{}
\groupno{}
\stuname{}
\expdate{\expdatefmt\today}
\expname{全息技术实验}

%%
%% Document body
%%

\begin{document}
% First page
% Some titles and personal information are defined in ``\maketitle''.
\maketitle

\section{预习}
简述全息照相的记录与再现原理。

\newpage

\section{原始数据记录}


\begin{table*}[ht]
    \renewcommand{\arraystretch}{1.4}
    \small\selectfont
    \centering
    \caption{光路信息}
    \begin{tabularx}{\textwidth}{|c|Y|Y|Y|Y|}\hline
        物光光强 & 参考光光强  & 物光光程(dm) & 参考光光程(dm)  & 参考光与物光的夹角(°) \\\hline
        1       &            &               &                  &                       \\\hline
        1       &            &               &                  &                       \\\hline
        1       &            &               &                  &                       \\\hline
        1       &            &               &                  &                       \\\hline
        1       &            &               &                  &                       \\\hline
    \end{tabularx}
\end{table*}

% Teacher signature
\makeatletter
\physicsreport@body@signature{preparation}
\makeatother

\newpage

\section{实验现象分析及结论}
试分析哪些因素会对全息成像有影响。
\vspace*{8cm}

\section{讨论题}
\begin{enumerate}
    \item 试比较全息照相与普通照相的异同点。
    \item 为什么用白光照射全息照片会出现彩带?为什么说观察到彩带即说明拍摄成功?
    \item 参考光与物光之间夹角的大小对成像有何影响?
\end{enumerate}

\end{document}